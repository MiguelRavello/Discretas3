\chapter{Algoritmo Lehmer GCD}

\section{Algoritmo}
Primero es importante definir una base sobre la cual se trabajará, tomaremos 1000 como base.
Esto significa que los dígitos de los números  a tratar serán agrupados en conjuntos de 3; por ejemplo,
si tenemos 657343812 y 431056703 se formarán grupos de (657)(343)(812) y (431)(056)(703), y se iterará 
sobre el grupo más significativo, sin importar que este grupo tenga 3, 2, ó 1 elemento, estos valores se almacenan en x\_ e y\_, pero nos aseguraremos que estas iteraciones se den siempre y cuando la cantidad de grupos restantes sea la misma (linea 1.3), si no x\_ y y\_ intercalarán valores por largo tiempo.
Así cada vez se trabaja sobre un x e y más pequeño, a su vez estos estarán disminuyendo a lo largo de la ejecución, cuando uno de ellos sea menor o igual a la base se podrá proceder con otro método de cálculo del gcd.\\
INPUT: Dos enteros positivos(x $\geq$ y) que sean $>=$ a la base.\\
OUTPUT: gcd(x, y)\\
\begin{itemize}
 \item[1] While y $\geq$ b do the following:
 \begin{itemize}
  \item[1.1] Set x\_, y\_ to be the high-order digit of x, y, respectively (y\_ could be 0)
  \item[1.2] $A\leftarrow1$, $B\leftarrow0$, $C\leftarrow0$, $D\leftarrow1$ 
  \item[1.3] If ($grupos\_x == grupos\_y$) esto se agregó al algoritmo del presente libro 
  \item[1.4] While ($y\_ + C$) $!= 0$ and $(y\_ + D)!= 0$ do the following:
  \begin{itemize}
   \item $q\leftarrow (x\_ + A)/(y\_ + C)$ , $q\_ \leftarrow (x\_ + B)/(y\_+ D)$ 
   \item if $q = q\_$ then go to step 1.5
   \item $t \leftarrow A-qC, A \leftarrow C, C \leftarrow t, t\leftarrow B - qD, B\leftarrow D, D\leftarrow t$ 
   \item $t\leftarrow x\_ - qy\_, x\_\leftarrow y\_, y\_\leftarrow t$
  \end{itemize}
  \item[1.5] If $B = 0$, then $T \leftarrow x mod y$, $x\leftarrow y, y\leftarrow T$ 
  \item otherwise, $T\leftarrow Ax + By, u\leftarrow Cx + Dy, x\leftarrow T , y\leftarrow u$ \\
 \end{itemize}
 \item[2] Compute $v = gcd(x, y)$ using Algorithm 2.104 
 \item[3] Return(v)
\end{itemize}

\section{Implementaci\'on}

Para determinar los grupos de cifras del número se ha empleado un arreglo que contiene las potencias de la base (1000, 1000, 1 000 000, 1 000 000 000,  1 000 000 000), con el primer y último valor repetido, para que con búsqueda binaria se determine el valor adecuado con el que se hará la división para determinar x\_ e y\_. Por ejemplo: Al buscar 12 345 678 en el arreglo se retornará el índice de 1 000 000 y simplemente se hace 12 345 678 / 1 000 000 para obtener el 12 buscado, el índice sirve para determinar la cantidad de grupos restantes, en este caso 2.
Una vez simplificado lo suficiente se emplea Dijkstra para los números reducidos.

\begin{lstlisting}[language=C++]
#include <iostream>
#include <limits.h>

#define MAX_POTENCIAS 3 // para generar un arreglo de potencias de la base

using namespace std;
using Tipo = unsigned int;

void lehmer_gcd(Tipo x, Tipo y);
Tipo dijkstra_euclides( Tipo a,  Tipo b);

int main()
{
    double segs;

    Tipo x = 657343812;
    Tipo y = 431056703;

    cout << "Aplicando Lehmer, gcd(" << x << ',' << y << ")\n";
    lehmer_gcd(x,y);

    int dj;
    clock_t t_ini = clock();
    dj = dijkstra_euclides(x,y);
    clock_t t_fin = clock();
    cout << dj << " en " << (double)(t_fin - t_ini)*1000.0 / CLOCKS_PER_SEC;
    return 0;
}

Tipo *genera_array_base(Tipo base);
Tipo digitos_base(Tipo num, Tipo arr[], Tipo &grupos);

void lehmer_gcd(Tipo x, Tipo y)
{
    clock_t t_ini = clock();

    Tipo x_, y_, a, b, c, d;
    Tipo q, q_, t, tt, u;
    Tipo grupos_x;
    Tipo grupos_y;
    Tipo base = 1000; // determina un arreglo de potencias de la base.
    Tipo length_bits = sizeof(int)*CHAR_BIT;
    Tipo *arr_potencias = genera_array_base(base); // se crea un array con las potencias de la base.
                                       // si es base 2 se trata de una forma especial (moviendo bits)
    while(y >= base){

        x_ = digitos_base(x, arr_potencias, grupos_x ); // x_ tendrá los dígitos más significativos que unidos serán <= a la base.
                              // la función usa búsqueda binaria en el array de potencias  de la base
        y_ = digitos_base(y, arr_potencias, grupos_y);
	
        a = 1; b = 0; c = 0; d = 1;
        if(grupos_x == grupos_y){ //necesario pues en la siguiente vuelta hay que asegurar que x e y tengan la misma cantidad d cifras
            while( ((y_+c) != 0) && ((y_+d) != 0) ){
                q = (x_+a) / (y_+c);
                q_ = (x_+b)/ (y_+d);
                if (q != q_){
                    break;
                }
                t=a-q*c;
                a = c;
                c = t;
                t = b - q*d;
                b = d;
                d = t;
                t = x_ - q*y_;
                x_ = y_;
                y_ = t;
            }
        }
        if (b == 0){
            tt = x%y;
            x = y;
            y = tt;
        }
        else{
            tt = a*x + b*y;
            u = c*x + d*y;
            x = tt;
            y = u;
        }
    }

    clock_t t_fin = clock();

    cout << "Lehmer redujo a gcd(" << x << ',' << y << ") en ";
    double segs = (double)(t_fin - t_ini) / CLOCKS_PER_SEC;
    cout << segs * 1000.0 << " milisegundos. \n" << '\n';

    cout << "Aplicando euclides(dijkstra): gcd(" << x << ',' << y << ") = ";

    t_ini = clock();
    Tipo v = dijkstra_euclides(x,y);
    t_fin = clock();
    cout << v << '\n';
    segs += (double)(t_fin - t_ini) / CLOCKS_PER_SEC;
    cout << "\nTiempo total:" << segs * 1000.0 << " milisegundos" << '\n';
}

Tipo *genera_array_base(Tipo base)
{
    //repetir el primer y último elemento para
    //  retornar lo deseado en la busq. binaria
    //  así se agregan dos elementos más
    Tipo *arr = new Tipo[MAX_POTENCIAS+2];

    arr[0] = base;
    Tipo potencia = 1;
    for(Tipo i = 1; i != MAX_POTENCIAS+1; i++){
        potencia *= base;
        arr[i] = potencia;
    }
    arr[MAX_POTENCIAS+1] = potencia;

    return arr;
}

Tipo b_binaria(Tipo num, Tipo arr[], Tipo low, Tipo high); // devuelve el valor de la
                                                    // potencia <= num dentro de arr
Tipo digitos_base(Tipo num, Tipo arr[], Tipo &grupos)
{
    grupos = b_binaria(num, arr, 0, MAX_POTENCIAS+2 -1);
    if (grupos == 0)
        grupos = 1;
    else
        if (grupos == MAX_POTENCIAS+1 )
            grupos = MAX_POTENCIAS;
    return num / arr [grupos];
}
Tipo b_binaria(Tipo x, Tipo arr[], Tipo low, Tipo high)
{
    Tipo medio;
    if (high > low)
        medio= (high-low)/2 + low;
    else
        medio = (low-high)/2 + low;
    if(arr[medio] == x || (low > high))
        return (low-1); //medio == low, asi retorna la potencia menor de la base
    else{
        if (arr[medio] < x)
            return b_binaria(x,arr,medio+1, high);
        else
            return b_binaria(x,arr,low,medio-1);
    }
}
Tipo dijkstra_euclides( Tipo a,  Tipo b){
    if(a==0 || b==0)
        return a+b;
    else if(a>b){
        a%=b;
        return dijkstra_euclides(a,b);
    }
    else if(b>a){
        b%=a;
        return dijkstra_euclides(a,b);
    }
}
\end{lstlisting}

\section{Seguimiento del algoritmo}

\begin{center}
\begin{sideways}
\begin{tabular}{|c|c|c|c|c|c|c|c|c|c|c|c|}
\hline
x&y&grupos\_x&grupos\_y&x\_&y\_&a&b&c&d&q&q\_ \\ \hline
657343812&431056703&2&2&657&431&1&0­&0&1&1&1 \\ \hline
& & & & & & 0&  1& 1& -1& & \\ \hline
& & & & 431& 226& 1& -1& -1& 2& 1& 1 \\ \hline
& & & & & & 1& -1& -1& 2& & \\ \hline
& & & & 226& 205& -1& 2& 2& -3& 1& 1 \\\hline
& & & & 205& 21& & & & & 8& 11 \\\hline
204769594& 21517515& 2& 2& 204& 21& 1& 0& 0& 1& 9& 9 \\\hline
& & & & & & 0& 1& 1& -9& & \\\hline
& & & & 21& 15& & & & & 1& 3 \\\hline
21517515& 11111959& 2& 2& 21& 11& 1& 0& 0& 1& 2& 1 \\\hline
11111959& 10405556& 2& 2& 11& 10& 1& 0& 0& 1& 1& 1 \\\hline
&&&&&&1&-1&-2&3&&\\\hline
&&&&191&133&&&&&1&1\\\hline
&&&&&&-2&3&3&-4&&\\\hline
&&&&133&58&&&&&2&2\\\hline
&&&&&&3&-4&-8&11&&\\\hline
&&&&58&17&&&&&6&1\\\hline
55553&23830&1&1&55&23&1&0&0&1&2&2\\\hline
&&&&&&0&1&1&-2&&\\\hline
&&&&23&9&&&&&2&3\\\hline
23830&7893&1&1&23&7&1&0&0&1&3&2\\\hline
7893&151&&&&&&&&&&\\\hline
\end{tabular}
\end{sideways}
\end{center}

  
    
      
      
      
      
    
      

