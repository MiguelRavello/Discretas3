\chapter{Algoritmo de Euclides clasico}

\section{Definici\'on}
El algoritmo de Euclides es un m\'etodo antiguo y eficaz para calcular el m\'aximo com\'un divisor (MCD). Fue originalmente descrito por Euclides en su obra Elementos.\\
El algoritmo de Euclides se basa en la aplicaci\'on sucesiva del siguiente lema:\\ 
Sean $\:a, b, q, r \:\in\:Z$ tales que: $a = bq + r$ con $\:b > 0\:$ y $\:0 ≤ r < b$. Entonces $\:mcd (a, b) = mcd (b, r)$
\section{Algoritmo}
Recordemos que $\:mod(a, b)$ denota el resto de la división de a por b. En este algoritmo, en cada paso $\:r = mod (rn+1, rn)$ donde $\:rn+1 = c$ es el dividendo actual y $\:rn = d$ es el divisor actual.
Luego se actualiza $\:rn+1 = d$ y $d = r$. El proceso continúa mientras d no se anule.\\
Datos: $\:a,b \in Z\:/b\neq0$\\
Salida: $mcd(a,b)$\\
\begin{equation*}
 \begin{align}
  c=&|a|,d=|b|;\\
  while&\:d\neq0\:do\\
  r&=mod(c,d);\\
  c&=d;\\
  d&=r;\\
  return&\:mcd(a,b)=|c|;
 \end{align}
\end{equation*}
\section{Seguimiento del codigo}
\begin{table}[!h]
\label{tablax}
\begin{center}
\begin{tabular}{|c|c|c|}
\hline 
i&a&b \\
\hline
1&4294967295&3294967290\\\hline
2&3294967290&1000000005\\\hline
3&1000000005&294967275\\\hline
4&294967275&115098180\\\hline
5&115098180&64770915\\\hline
6&64770915&50327265\\\hline
7&50327265&14443650\\\hline
8&14443650&6996315\\\hline
9&6996315&451020\\\hline
10&451020&231015\\\hline
11&231015&220005\\\hline
12&220005&11010\\\hline
13&11010&10815\\\hline
14&10815&195\\\hline
15&195&90\\\hline
16&90&15\\\hline
17&15&0\\\hline
\end{tabular}
\end{center}
\caption{Seguimiento del codigo}
\end{table}

\begin{lstlisting}[language=C++]
 ZZ gcd(ZZ &a, ZZ &b){                         
         if (b == 0)                             
           return a;                             
         a%=b;                                   
         return gcd(b, a);                       
 }   
\end{lstlisting}
